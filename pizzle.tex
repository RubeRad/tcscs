\documentclass[12pt]{article}
%\usepackage{geometry}
%\geometry{letterpaper, portrait, margin=1cm}
\begin{document}
\title{Pizza Puzzle}
%\maketitle
\pagestyle{empty}

\section*{A Pizza Puzzle}
8 roommates are going to order one giant pizza, and they're having trouble
reaching consensus about the 5 available toppings:
\begin{itemize}
\item Mushrooms
\item Sausage
\item Ham
\item Pineapple
\item Anchovies
\end{itemize}

They agree that everybody gets to state {\bf TWO} preferences (positive or
negative), and that they'll get a pizza if they can find a combination of
toppings that satisfies at least {\bf ONE} of everybody's preferences.

\begin{enumerate}
\item Eddie is easy-going. As long as there's either Mushrooms or Sausage, he's
  good. Eddie's preferences are $(m \lor s)$.
\item Helen {\sc hates} Hawaiian pizza. If there's no ham, or if there's no
  pineapple (even if the other is present), then technically it's not Hawaiian,
  so Helen would be happy. Helen's preferences are $(\neg h \lor \neg p)$.
\item Jared's preferences are not sausage, or yes pineapple: $(\neg s \lor p)$
\item I didn't feel like making up 5 more names, but you get the point.
\end{enumerate}

The agreed-upon rule of satisfying at least one preference for each roommate looks like this:
$$(m \lor s) \land
(\neg h \lor \neg p) \land
(\neg s \lor p) \land
(\neg m \lor s) \land
(\neg h \lor a) \land
(\neg h \lor \neg m) \land
(\neg p \lor \neg s) \land
(\neg a \lor \neg m)$$

Can the roommates be satsfied?

\vspace{1cm}
How about these very picky roommates, that don't seem to like anything?
$$(\neg m \lor \neg p) \land
(\neg s \lor \neg m) \land
(\neg p \lor m) \land
(\neg s \lor m) \land
(\neg h \lor \neg s) \land
(\neg p \lor \neg a) \land
(a \lor \neg p) \land
(\neg h \lor \neg p)$$

\clearpage
\section*{Homework}
For each of these two 2SAT problems:
\begin{enumerate}
\item
$(h \lor p) \land (m \lor s) \land (s \lor p) \land (a \lor \neg m)
  \land (h \lor a) \land (\neg p \lor h) \land (h \lor \neg s) \land
  (\neg s \lor p)$
\item
$(\neg a \lor p) \land (\neg s \lor a) \land (\neg a \lor \neg h) \land
  (\neg h \lor a) \land (p \lor \neg s) \land (\neg m \lor a) \land (\neg
  m \lor \neg a) \land (h \lor m)$
\end{enumerate}

Do the following:
\begin{itemize}
\item At {\tt https://graphonline.ru/en}, click Graph and Import from File, and (from
  Google Classroom) {\tt pizza5.graphml}.
\item Create two arcs for every clause, according to these principles:
  \begin{itemize}
  \item If it's not the first one $\Rightarrow$ it must be the second one.
  \item If it's not the second one $\Rightarrow$ it must be the first one.
  \end{itemize}
\item Save your progress with Graph/Export to File. Name it using 1 or 2 for
  the problem number, and your own name {\tt 2SAT[12]arcs[NAME].json}.
\item (Try to) rearrange the nodes so arcs flow from left to right, Graph/Export
  your rearrangement as {\tt 2SAT[12]flow[NAME].json}.
\item If there is a cycle that prevents a satisfying assignment, specify it,
  like ``$a \Rightarrow \neg b \Rightarrow \neg a \Rightarrow c \Rightarrow a$,
  thus $a\Rightarrow\neg a$ and $\neg a\Rightarrow a$, so there is no possible
  truth value for $a$.''
\item If there is a satisfying assignment,
  \begin{itemize}
    \item Construct it right-to-left, something like, ``$a$ is a sink node, set
      it to {\sc true} (thus $\neg a$ is {\sc false}). Next $\neg b$ is set to
      {\sc true} (so $b$ is {\sc false})... and now all values are assigned.''
    \item Plug the assignments into the original 2SAT statement and demonstrate
      it evaluates to {\sc true}.
  \end{itemize}

\item Submit your saved .graphml files (4 of them) along with your solutions.
\item If you find it helpful, you can screenshot/print your logical flow
  diagrams, and write/draw on them to provide the solutions, for instance
  highlight a cycle and the conflicting nodes on the cycle, or annotate the
  order of truth assignments.
\end{itemize}

\end{document}
